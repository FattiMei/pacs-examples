\documentclass[10pt]{beamer}
\usetheme{default}
\setbeamercovered{transparent}
\setbeamertemplate{navigation symbols}{}
\setbeamertemplate{footline}{
    \flushright{\hfill \insertframenumber{}/\inserttotalframenumber}
}

\begin{document}
    \title{Functional}
    \author{Pasquale Claudio Africa}
    \date{27/03/2020}
    
\begin{frame}
    \maketitle
\end{frame}

\begin{frame}{Exercise 1 - Horner algorithm}
Starting from the (incomplete) example for the Horner algorithm:
\begin{enumerate}
\item Implement the \texttt{eval()} and \texttt{eval\_horner()} functions to compute:
\begin{align*}
p_\text{eval}(x) &= a_0 + a_1x + a_2x^2 + a_3x^3 + \ldots + a_nx^n, \\
p_\text{Horner}(x) &= a_0 + x \left(a_1 + x \left(a_2 + x \left(a_3 + \ldots + x\left(a_{n-1} + x \, a_n\right) \ldots \right) \right) \right).
\end{align*}
\item Implement an \texttt{evaluate\_poly()} function by manually looping over the input points.
\item Modify \texttt{evaluate\_poly()} to use \texttt{std::transform}.
\item Implement an \texttt{evaluate\_poly\_parallel()} that makes use of the parallel execution policies of \texttt{std::transform} (available since \texttt{C++17}).
\item Convert \texttt{eval} and \texttt{horner} from function pointers to \texttt{std::function}.
\item (Optional) Let the user choose from command line the type of evaluation to run (standard, Horner) and the parallelization policy (sequential, parallel).
\end{enumerate}

Note: the parallel version requires to link against the Intel Threading Building Blocks (\texttt{TBB}) library.
\end{frame}

\begin{frame}{Exercise 2 - Newton solver}
\begin{enumerate}
\item Implement a \texttt{newton\_solver} class.
\item Let algorithm parameters be read from the command line using \texttt{GetPot}.
\item Write different \texttt{main} files that pass functions and derivatives as:
\begin{enumerate}
\item function pointers;
\item lambda functions;
\item (Optional) \texttt{muParser} functions (the \texttt{muParser} library can be installed by running \texttt{./install\_PACS.sh} in \texttt{\${PACS\_ROOT}/Extras/muParser}).
\end{enumerate}
\item Use the solver to solve the equation
\begin{equation*}
x^3 + 5 x + 3 = 0.
\end{equation*}
\end{enumerate}
\end{frame}

\end{document}

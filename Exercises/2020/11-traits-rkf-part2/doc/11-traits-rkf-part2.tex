\documentclass[10pt]{beamer}
\usetheme{default}
\setbeamercovered{invisible}
\setbeamertemplate{navigation symbols}{}
\setbeamertemplate{footline}{
    \flushright{\hfill \insertframenumber{}/\inserttotalframenumber}
}

\usepackage{listings}

% User-defined colors.
\definecolor{DarkGreen}{rgb}{0, .5, 0}
\definecolor{DarkBlue}{rgb}{0, 0, .5}
\definecolor{DarkRed}{rgb}{.5, 0, 0}
\definecolor{LightGray}{rgb}{.95, .95, .95}
\definecolor{White}{rgb}{1.0,1.0,1.0}
\definecolor{darkblue}{rgb}{0,0,0.9}
\definecolor{darkred}{rgb}{0.8,0,0}
\definecolor{darkgreen}{rgb}{0.0,0.85,0}

% Settings for listing class.
\lstset{
  language=C++,                        % The default language
  basicstyle=\small\ttfamily,          % The basic style
  backgroundcolor=\color{White},       % Set listing background
  keywordstyle=\color{DarkBlue}\bfseries, % Set keyword style
  commentstyle=\color{DarkGreen}\itshape, % Set comment style
  stringstyle=\color{DarkRed}, % Set string constant style
  extendedchars=true % Allow extended characters
  breaklines=true,
  basewidth={0.5em,0.4em},
  fontadjust=true,
  linewidth=\textwidth,
  breakatwhitespace=true,
  showstringspaces=false,
  lineskip=0ex, %  frame=single
}

\begin{document}
    \title{A Runge-Kutta-Fehlberg solver\protect\\ using traits and \texttt{Eigen} (part 2)}
    \author{Pasquale Claudio Africa}
    \date{04/06/2020}

\begin{frame}[plain, noframenumbering]
    \maketitle
\end{frame}

\begin{frame}{Homeworks}
This folder contains the solution to the homeworks of Lab 10.
\begin{enumerate}
\item Use the provided solver to solve the SIR model presented in \url{https://arxiv.org/pdf/2003.00122.pdf} (more realistic coefficients can be found \href{https://arxiv.org/pdf/2003.14391.pdf}{\textcolor{title.fg}{here}}).
\item Implement the Fehlberg12 and the Dormand-Prince methods (\url{https://en.wikipedia.org/wiki/List_of_Runge\%E2\%80\%93Kutta_methods\#Embedded_methods})
\item Modify the constructor of the \texttt{RKF} class so that the chosen RKF method has a default value.

\item Define a class to handle the input options for the \texttt{RKF} class, and provide the corresponding setter/getter methods and possibly a method to parse options from file.

\item (Advanced) Implement the factory pattern to the \texttt{ButcherArray} class, so that the actual method can be chosen dynamically exploiting polymorphism.
\end{enumerate}
\end{frame}

\end{document}

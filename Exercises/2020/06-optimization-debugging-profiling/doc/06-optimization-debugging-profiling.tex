\documentclass[10pt]{beamer}
\usetheme{default}
\setbeamercovered{invisible}
\setbeamertemplate{navigation symbols}{}
\setbeamertemplate{footline}{
    \flushright{\hfill \insertframenumber{}/\inserttotalframenumber}
}

\usepackage{listings}

% User-defined colors.
\definecolor{DarkGreen}{rgb}{0, .5, 0}
\definecolor{DarkBlue}{rgb}{0, 0, .5}
\definecolor{DarkRed}{rgb}{.5, 0, 0}
\definecolor{LightGray}{rgb}{.95, .95, .95}
\definecolor{White}{rgb}{1.0,1.0,1.0}
\definecolor{darkblue}{rgb}{0,0,0.9}
\definecolor{darkred}{rgb}{0.8,0,0}
\definecolor{darkgreen}{rgb}{0.0,0.85,0}

% Settings for listing class.
\lstset{
  language=C++,                        % The default language
  basicstyle=\small\ttfamily,          % The basic style
  backgroundcolor=\color{White},       % Set listing background
  keywordstyle=\color{DarkBlue}\bfseries, % Set keyword style
  commentstyle=\color{DarkGreen}\itshape, % Set comment style
  stringstyle=\color{DarkRed}, % Set string constant style
  extendedchars=true % Allow extended characters
  breaklines=true,
  basewidth={0.5em,0.4em},
  fontadjust=true,
  linewidth=\textwidth,
  breakatwhitespace=true,
  showstringspaces=false,
  lineskip=0ex, %  frame=single
}

\begin{document}
    \title{Optimization, debugging and profiling}
    \author{Pasquale Claudio Africa}
    \date{24/04/2020}

\begin{frame}[plain, noframenumbering]
    \maketitle
\end{frame}

\begin{frame}{Exercise 1}

Implement a
library for matrices (and cloumn vectors 
implemented as 1-column matrices) based on 
STL containers such as \lstinline[language=C++]{std::vector<>}
or \lstinline[language=C++]{std::array<>}
with the following methods/functions
\begin{itemize}
\item transpose : $A = A^{T}$
\item solve : solve $A x = b$ by means of Gaussian eliminition with partial pivoting by rows
\item operator* : matrix-matrix and matrix-vector multiplication
\end{itemize}
\end{frame}

\begin{frame}[allowframebreaks]{Solution}
\begin{itemize}
\item \lstinline[language=C++]{std::array<>}
      requires the size to be known at compile time,
      in order to make the \lstinline[language=C++]{matrix}
      object configurable at run time we use 
      \lstinline[language=C++]{std::vector<>}\\[3mm]
\item matrices ar organized as \emph{column major}, {\it i.e.}
      %$A(i, j) = $ \lstinline[language=C++]{data[i + j * rows ()]}, 
      conversion from 1d to 2d indexing is performed by the utility
      (private) function \lstinline[language=C++]{sub2ind}\\[3mm]
\item access to elements is implemented both in const and non-const
      versions, by overloading \lstinline[language=C++]{operator()} \\[3mm]
\item data is private, \emph{getter methods} expose what is needed to 
      the user, both const and non-const versions are provided \\[3mm]
\item naive implementation of matrix-matrix multiplication is slow 
      because it has low \emph{data locality}
      (see below more about memory hierarchy and cache optimized algorithms), 
      simply transposing the left matrix factor improves performance significantly\\[3mm]
\item \lstinline[language=C++]{\#include <ctime>} header provides timing utilities,
      \lstinline[language=C++]{tic ()} and \lstinline[language=C++]{toc (x)} macros
      start and stop the timer (like in Matlab)
\end{itemize}
\end{frame}

\begin{frame}{Exercise 1.2}

\begin{itemize}
\item Transpose the first factor in matrix multiplication before performing the product
\item Compare speed with previous implementation
\end{itemize}
\end{frame}

\begin{frame}{Exercise 1.3}

\begin{itemize}
\item Include the {\tt Eigen/Dense} header
\item Use the {\tt Eigen::Map} template class to wrap the matrix data and interpret it as {\tt Eigen::MatrixXd}
\item Compare speed with previous implementation
\end{itemize}
\end{frame}

\begin{frame}{Exercise 2}
The program \texttt{integer-list} in the directory \texttt{02-bug} has:

\begin{itemize}
    \item One compile error.
    \item One run-time error.
    \item One memory leak.
    \item A potential memory leak that is not captured by the main.
\end{itemize}

Find all the issues and fix them. \\
The directory \texttt{02-bug-solution} contains the fixed code,
please don't look at it before trying to solve the exercise by yourself.
\end{frame}

\end{document}

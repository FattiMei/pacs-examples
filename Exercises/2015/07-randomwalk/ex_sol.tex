\section*{Solution}

\subsection*{Exercise 1}

The solution is spread in different files. The first we look at is
\texttt{Util.hpp}
%
\lstset{basicstyle=\scriptsize\sf}
\lstinputlisting[caption=\texttt{Util.hpp} file.]
    {./src/ex1/Util.hpp}
\lstset{basicstyle=\sf}
%
The \cpp{Rescaler} is a simple utility that maps an integer number in
$[0,\ldots,N)$ in an integer number in $\{-1,0,1\}$. This class will be used to
translate the integer numbers generated by the random engine in	velocities for
the set of particles.

The \cpp{oflag_T} is a \cpp{typedef} that is useful to declare the different way
to output in our routines. The approach that we use here uses bit-flags, and is
simila to the way that input/output flags are managed in the STL. each option is
represented by a single bit inside a wider integer variable. Afterwards, all
possible combinations of options can be built using bit-wise logical operators.
In a similar way also the \cpp{std::bitset} can be used.

The second file we look at is \texttt{Distributions.hpp}
%
\lstset{basicstyle=\scriptsize\sf}
\lstinputlisting[caption=\texttt{Distibutions.hpp} file.]
    {./src/ex1/Distributions.hpp}
\lstset{basicstyle=\sf}
%
This file stores a locally implemented distribution called
\cpp{FakeDistribution}. As the name says, this distribution is in fact
fictitious, since it gives exactly equal percentages to each value in the range,
in fact this distribution is meant as a testing for the \cpp{RandomWalk} class.
In order to use it, it must have an identical inteface to the
\cpp{discrete_distribution} class in the STL, at least for the methods that are
used in the \cpp{RandomWalk} class, in particular it must have a constructor
that takes an \cpp{initializer_list} as an argument. 

The main class \cpp{RandomWalk} is implemented in the \texttt{RandomWalk.hpp} 
file that follows
%
\lstset{basicstyle=\scriptsize\sf}
\lstinputlisting[caption=\texttt{RandomWalk.hpp} file.]
    {./src/ex1/RandomWalk.hpp}
\lstset{basicstyle=\sf}
%
In order to work with different kind of distribution, the class is templated on
the distribution type. The implementation is then straightforward, making use of
the object oriented philosophy of the blocks that build the final product. We
note the use of the output bit flag to decide which sections of code should be
run.

The main file is minimal
%
\lstset{basicstyle=\scriptsize\sf}
\lstinputlisting[caption=\texttt{main\_randomlap.cpp} file.]
    {./src/ex1/main_randomlap.cpp}
\lstset{basicstyle=\sf}
%
and has the version with the \cpp{FakeDistribution} commented. The use of
different weights or a weight also for particles that do not move is achieved by
simply passing the correct initializer list that represents the weights.

\subsection*{Exercise 2}

The implementation of a MonteCarlo method to compute integrals can be seen in
the Examples section.

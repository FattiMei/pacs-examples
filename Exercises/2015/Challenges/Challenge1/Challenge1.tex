%\documentclass[handout]{beamer}
\documentclass{beamer}
\usetheme{default}
\usepackage[english]{babel}
\usepackage{pgfpages}
%\pgfpagelayout{2 on 1}[a4paper]


%
% The following info should normally be given in you main file:
%
\usepackage{generalsetting}

\setbeamercovered{transparent}

\renewcommand{\lezione}{Challenges}
\begin{document}
\begin{frame}{Challenge 1}
\alert{Difficulty: low}
Create two simple structures or classes, \li!Edge! and \li!Triangle! 
storing two  and three unsigned integers that represent the identifiers of the points
defining an edge and a triangle, respectively. 

Given a container of triangles that represents a mesh (a simple one, initialized by hand), create a function
accepting the container as input and returning a \li!set! containing all edges, without repetition.

\end{frame}

\begin{frame}{Challenge 2}
\alert{Difficulty: medium}
From the same structures created in Challenge1. Create a function
that accepts the \li!Triangle! container as input and returns a set containing only the \textcolor{blue}{boundary edges}.
\smallskip

\alert{Hint:} The triangles must be oriented consistently, for instance counter clockwise orientation.
\end{frame}

\begin{frame}{Challenge 3}
  \alert{Difficulty: medium-high} Repeat Challenges 1 and 2 for
  the classes \li!Tetrahedron! and \li!Triangle!. Create a container for the
  tetrahedra representing a mesh and find all faces and all boundary
  faces.  Do it for a very simple mesh that you can create by hand:
  two tetrahedra are enough!  
\smallskip

  \alert{Hint:} The tetrahedra must be oriented consistently, for
  instance following the right-hand rule.
\end{frame}



\end{document}

%%% Local Variables: 
%%% mode: latex
%%% TeX-master: t
%%% End: 

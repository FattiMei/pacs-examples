 %&latex
\documentclass[smaller,a4paper]{beamer}
\usepackage{amsmath,amssymb,pdfsync,listings}
\usepackage{verbatim}
\usepackage{graphicx}
\usepackage{truncate}
%%\usepackage{mpmulti}
\usepackage{times}
\lstset{
  language=[ISO]C++,                        % The default language
  basicstyle=\small,                   % The basic style
  %backgroundcolor=\color{LightGray},       % Set listing background
  %keywordstyle=\color{DarkBlue}\bfseries,  % Set keyword style
  %commentstyle=\color{DarkGreen}\itshape,  % Set comment style
  %stringstyle=\color{DarkRed},             % Set string constant style
  extendedchars=true                        % Allow extended characters
}
\newcommand{\cpp}[1]{\lstinline!#1!}
\usepackage[english]{babel}

\begin{document}
\title{Using STL Containers to Implement a Sparse Matrix Class}
\frame{\titlepage}

%---------------------------------------------------------------------------------

\begin{frame}[fragile]
\    \frametitle{Sparse Matrices}
\begin{itemize}
\item A sparse (square) matrix is a matrix that has a number of non-zero coefficients proportional to $n$ rather than $n^{2}$
\item It is not convenient to store all entries including zeros
\item It is not convenient to include zero entries when doing algebra with sparse matrices
\end{itemize}
\end{frame}

\begin{frame}
    \frametitle{A Sparse Matrix Class Based on STL Containers}
\begin{itemize}
\item Consider the following class:\\[3mm]
\cpp{class sparse_matrix : public std::vector<std::map<unsigned int, double> >}
\item What happens if we do:\\[3mm]
\cpp{A = sparse_matrix (4); 
     auto x = A[2][2] }
\item What happens if we do:\\[3mm]
\cpp{A = sparse_matrix (4); 
     A[2][2] = 1.0;}

\item Cool! This features make \cpp{class sparse_matrix : public std::vector<std::map<unsigned int, double> >}
      a great candidate for the implementation of a (row oriented) sparse matrix!
\end{itemize}
\end{frame}

\begin{frame}\frametitle{Useful things to know about the \cpp{std::map} container}
\begin{itemize}

\item \cpp{mapped_type& std::map::operator[] (const key_type& k);}, \\
A call to this function is equivalent to: \\
\cpp{(*((this->insert(make_pair(k,mapped_type()))).first)).second}

\item A similar member function, \cpp{std::map::at}, has the same 
      behavior when an element with the key exists, but throws an 
      exception when it does not.
      
\item \cpp{iterator find (const key_type& k);} \\
      \cpp{const_iterator find (const key_type& k) const;}\\
      Searches the container for an element with a key equivalent to \cpp{k} and returns 
      an iterator to it if found, otherwise it returns an iterator to \cpp{map::end ()}.

\item \cpp{size_type count (const key_type& k) const;} \\
Searches the container for elements with a key equivalent to k and returns the number of matches.
Because all elements in a map container are unique, the function can only return 1 (if the element is found) or zero (otherwise).

\end{itemize}
\end{frame}

\begin{frame}[fragile,allowframebreaks]
\frametitle{A recap example about inheritance}
\begin{lstlisting}
#include <iostream>

class A_t
{
public:
  void 
  A_m () { std::cout << "calling A_t::A_m" << std::endl; };
  
  virtual 
  void 
  B_m () { std::cout << "calling A_t::B_m" << std::endl; };

};

class B_t :
  public A_t
{
public:
  void 
  B_m () { std::cout << "calling B_t::B_m" << std::endl; };

  void 
  A_m () { std::cout << "calling B_t::A_m" << std::endl; };
};

int 
main (void)
{
  B_t b;
  A_t *a = &b;

  // b is of class B_t, derived from A_t 
  // therefore it has method A_m definide in B_t
  // but also the implementation defined in A_m
  b.A_m ();
  b.A_t::A_m ();

  // Also A_t::B_m even though it is virtual can be
  // accessed via a qualifier
  b.A_t::B_m ();

  // a is a pointer to an object of class  A_t
  // if I invoke method A_m the compiler chooses
  // the implementation given in A_t
  a->A_m ();

  // to access the method B_t::A_m I need to use a cast
  static_cast<B_t*> (a) -> A_m ();

  // If I invoke method B_m through
  // the pointer a, instead, the compiler chooses B_t::B_m
  // instead of A_t::B_m because A_t::B_m is virtual
  a->B_m ();

  // to access method A_t::A_m I need to add a qualifier
  a->A_t::A_m ();

  return 0;
};
\end{lstlisting}

\end{frame}

\begin{frame}\frametitle{Exercises}
\begin{itemize}
\item Implement a libary providing a \cpp{sparse_matrix} class inheriting 
      from \cpp{std::vector<std::map<unsigned int, double> >} \\[2mm]
\begin{itemize}
\item A basic implementation is already given as a starting point \\[2mm]
\item Write the makefile \\[2mm]
\item Adapt \cpp{fem1d} to use this class \\[2mm]
\end{itemize}
\item Create an abstract interface class \cpp{general_matrix} and derive 
both \cpp{matrix} and \cpp{sparse_matrix} from it \\[2mm]
\item Change \cpp{fem1d} so that the type of matrix can be selected by the user \emph{at run time}. 
\end{itemize}

\end{frame}




\end{document}
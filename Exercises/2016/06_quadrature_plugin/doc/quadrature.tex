 %&latex
\documentclass[smaller,a4paper]{beamer}
\usepackage{amsmath,amssymb,pdfsync,listings}
\usepackage{graphicx}
\usepackage{truncate}
%%\usepackage{mpmulti}
\usepackage{times}

\newcommand{\largenameref}[1]{\textbf{\huge{\nameref{#1}}}}
\newcommand{\largehyperlink}[2]{\textbf{\Large\hyperlink{#1}{#2} }}
\newcommand{\largehypertarget}[2]{\textbf{\color{red} \hypertarget{#1}{#2} }}
\newcommand{\Largehypertarget}[2]{\textbf{\Large \color{red} \hypertarget{#1}{#2} }}
\newcommand{\Int}[2]{\displaystyle{\int\limits_{#1}^{#2}}}      
\newcommand{\Sum}[2]{\displaystyle{\sum\limits_{#1}^{#2}}}      
\newcommand{\Myfoilheadskip}[1]{\begin{frame}\frametitle{#1}}
\newcommand{\trace}[2]{\left. #1 \right\rvert_{_{#2} }  }
\newcommand{\mtrx}[1]{\mathbf{#1}}
\newcommand{\vect}[1]{\mathbf{#1}}
\newcommand{\abs}[1]{\left|#1\right|}

\usepackage[english]{babel}
\lstset{
  language=[ISO]C++,                       % The default language
  basicstyle=\scriptsize,                  % The basic style
  extendedchars=true                       % Allow extended characters
}
\newcommand{\cpp}[1]{\lstinline!#1!}

\begin{document}
\title{Galerkin/Linear Finite Elements Method in 1d, with generic quadrature}
\frame{\titlepage}


\begin{frame}\frametitle{Exercise}
\begin{itemize}
\item adapt the fem1d code to allow the user to specify the quadrature rule as the name of a dynamically loadable object
\begin{itemize}
\item the dynamically loadable object should define a function named \cpp{integrate}
\item \cpp{double integrate (std::function<double (double)>, double a, double b)}
\end{itemize}
\item implement plugins for midpoint, trapezoidal an Simpson's rule
\item implement a plugin for quadrature with adaptive refinement
\end{itemize}
\end{frame}



\begin{frame}[fragile]
\frametitle{Implementation in {\tt C++} }
\only<1-2>{file {\tt fem1d.h}}
\only<3-8>{file {\tt fem1d.cpp}}
\only<9>{file {\tt adaptive\_quadrature.h}}
\tiny
\only<1>{\lstinputlisting[language=C++, firstline=1, firstnumber=1, lastline=35, numbers=left, numberstyle=\tiny\color{gray}]{../fem1d-0.8/fem1d.h}}
\only<2>{\lstinputlisting[language=C++, firstline=35, firstnumber=35, lastline=70, numbers=left, numberstyle=\tiny\color{gray}]{../fem1d-0.8/fem1d.h}}
\only<3>{\lstinputlisting[language=C++, firstline=1, firstnumber=1, lastline=35, numbers=left, numberstyle=\tiny\color{gray}]{../fem1d-0.8/fem1d.cpp}}
\only<4>{\lstinputlisting[language=C++, firstline=35, firstnumber=35, lastline=70, numbers=left, numberstyle=\tiny\color{gray}]{../fem1d-0.8/fem1d.cpp}}
\only<5>{\lstinputlisting[language=C++, firstline=70, firstnumber=70, lastline=105, numbers=left, numberstyle=\tiny\color{gray}]{../fem1d-0.8/fem1d.cpp}}
\only<6>{\lstinputlisting[language=C++, firstline=105, firstnumber=105, lastline=140, numbers=left, numberstyle=\tiny\color{gray}]{../fem1d-0.8/fem1d.cpp}}
\only<7>{\lstinputlisting[language=C++, firstline=140, firstnumber=140, lastline=175, numbers=left, numberstyle=\tiny\color{gray}]{../fem1d-0.8/fem1d.cpp}}
\only<8>{\lstinputlisting[language=C++, firstline=175, firstnumber=175, lastline=202, numbers=left, numberstyle=\tiny\color{gray}]{../fem1d-0.8/fem1d.cpp}}
\end{frame}



\end{document}
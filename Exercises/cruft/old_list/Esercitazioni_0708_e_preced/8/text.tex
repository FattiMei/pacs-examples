\section*{Es. 1}
Data una funzione $f(x)$ integrabile sull'intervallo $[a,b]$, lo scopo
di un integratore numerico � quello di fornire
un'approssimazione dell'integrale
\[
I_f(a,b) = \int_a^b f(x) dx.
\]

Partendo dalla traccia fornita,
\begin{enumerate}
\item considerare un integratore che utilizzi la regola di
  Cavalieri-Simpson su ogni sottointervallo. Posti:
\[
a=-3,\ b=4,\ f(x)=arctan(10x),
\]
calcolare un'approssimazione di $I_f(a,b)$.
Sapendo, inoltre, che la funzione
\[
F(x) = xarctan(10x) - \dfrac{log(1+100x^2)}{20}
\]
� una primitiva di $f$, verificare la correttezza dei risultati
ottenuti (al variare del numero di sottointervalli considerato).

\item Lo schema adottato � indipendente dalla regola di integrazione
  utilizzata in ciascun sottointervallo. Si modifichi opportunamente
  l'implementazione dell'integratore parametrizzandola
  rispetto a tale regola e si verifichi come varia la bont�
  dell'approssimazione utilizzando le regole di quadratura di
  Cavalieri-Simpson e del punto medio.

\emph{Suggerimento:} creare una classe \cpp{QuadratureRule} da passare
come parametro all'integratore. Questa soluzione pu� essere
implementata con la tecnica dei \cpp{templates}.

\end{enumerate}

\section*{Es. 2}
Il problema della ricerca delle radici di una funzione non lineare
$f:[a,b] \rightarrow \mathbb{R}$ pu� sempre essere trasformato in un
problema equivalente di ricerca del \emph{punto fisso} di una funzione
ausiliaria $\phi$:
\[
f(x) = 0\ \Longleftrightarrow\ x - \phi(x) = 0
\]

$\phi:[a,b] \rightarrow \mathbb{R}$ deve essere scelta in modo tale che $\phi(\alpha)=\alpha$ ogni
volta che $f(\alpha)=0$. L'algoritmo iterativo
\[
x^{(k+1)} = \phi( x^{(k)} ),\ k \geq 0,
\]
partendo da $x^{(0)}$ converge ad $\alpha$ sotto opportune ipotesi
\cite[\extref{6}]{Quarteroni.Sacco.ea:2000}.

Implementare il metodo di punto fisso di Newton, per il quale
\[
\phi_{Newt}(x) = x - \dfrac{f(x)}{f'(x)}.
\]
Considerata la funzione $f(x)=cos^2(2x)-x^2$ nell'intervallo $0 \leq x
\leq 1.5$, se ne calcoli lo zero con una tolleranza pari ad $\epsilon
= 10^{-10}$.

\emph{Suggerimento:} implementare il metodo di punto fisso come classe
derivata di una classe \cpp{FixedPointBase}; creare una classe
\cpp{SolveFP}, parametrizzata rispetto al metodo di punto fisso, per
il calcolo delle iterate successive.

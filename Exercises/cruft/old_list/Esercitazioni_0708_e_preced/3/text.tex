Sia $f\in C^{1}\left(a, b\right)$ una funzione reale che si annulla in
un punto $\alpha\in (a, b)$. Un algoritmo robusto per la
ricerca dello zero di tale funzione pu\`o essere costruito combinando
un metodo di basso ordine per cui sia garantita la convergenza globale
con uno di alto ordine che, partendo da un valore iniziale
sufficientemente vicino allo zero $\alpha$, si avvicini rapidamente ad
una sua approssimazione entro una fissata tolleranza $\epsilon$. Si
chiede di:
\begin{enumerate}
\item implementare un metodo per la ricerca dello zero di $f$
  sfruttando l'algoritmo della bisezione per avvicinarsi alla radice
  ed il metodo di Newton per ottenerne una stima accurata (si rimanda
  a \cite{Quarteroni.Sacco.ea:2000} per l'analisi).
\item Utilizzare il codice costruito per calcolare lo zero della
  funzione 
  \begin{equation*}
    f(x) = x^2 - 0.5
  \end{equation*}
  nell'intervallo $(0, 1)$. Si verifichi che, scegliendo come
  punto di partenza $x^{\left(0\right)} = 0$, il metodo di Netwon non
  converge. 
%\item Sovraccaricare l'operatore di scorrimento in modo che, inviando
%  ad uno \emph{stream} la classe corrispondente a ciascun metodo, si
%  ottenga una descrizione dell'istanza (nome del metodo, valori dei
%  parametri, numero di iterazioni necessarie per raggiungere la
%  convergenza all'ultima chiamata).
\end{enumerate}

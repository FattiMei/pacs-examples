\section*{Es. 1}
Pu\`o accadere, programmando, di volere utilizzare librerie scritte in
un linguaggio diverso. In questa esercitazione vedremo come sia
possibile utilizzare delle \emph{routine} Fortran all'interno di un
codice C++.
Nell' esercitazione 2 il problema derivante dalla discretizzazione
alle DF dell'equazione del calore era stato risolto mediante il metodo
di Gauss-Seidel, che non richiede la costruzione esplicita
dell'operatore discreto. Si chiede di modificare il codice in modo
che, dopo aver costruito e memorizzato l'operatore discreto, si
fattorizzi e risolva il sistema lineare ad esso associato mediante le
\emph{routine} Lapack \cpp{dgttrf} e \cpp{dgttrs}. Per memorizzare la
matrice tridiagonale, il vettore dei termini noti e quello delle
incognite si utilizzino dei vettori STL. 

Il codice cos\`i ottenuto ha ancora alcune limitazioni. In
particolare, per ripetere il calcolo con valori diversi dei parametri
fisici si dovrebbe modificare il listato e ricompilare. Ci\`o non solo
sarebbe poco pratico, ma rischioso qualora a modificare il codice sia
un utente che non conosce il C++.
D'altro canto, l'inserimento da terminale di tutti i dati del problema
risulterebbe tedioso, specie se solo pochi di essi cambiano ad ogni
esecuzione. Una buona alternativa \`e costituita dall'\emph{input} da
\emph{file}. Si richiede di modificare il codice in modo che i valori
dei parametri del problema vengano letti da un \emph{file} facendo uso
della libreria GetPot (\texttt{http://getpot.sourceforge.net}). 


\section*{Es. 2}

A partire dal testo dell'esercitazione 6, si consideri il caso in cui
il file di descrizione della mesh contenga anche l'indicazione del
tipo di condizione al bordo associata a ciascun nodo.
\lstset{basicstyle=\scriptsize\sf}
\lstinputlisting[caption=Una semplice \emph{mesh}: sono indicate le
condizioni al bordo associate ai nodi]{./es/2/mesh.msh}
\lstset{basicstyle=\sf}

Si utilizzi una opportuna struttura dati per associare
l'identificativo di ciascun nodo con la stringa che definisce la condizione
al bordo: si modifichi l'operatore di reindirizzamento ad output della
classe \cpp{Mesh} per riportare a schermo la condizione al bordo
associata a ciascun nodo.

Si implementino inoltre le strutture dati e i metodi necessari per
restituire a schermo una lista di nodi associati ad una certa stringa
(o un messaggio di avviso se la stringa non � associata ad alcun
nodo).

Per questo esercizio si sfruttino gli algoritmi ed i contenitori STL.

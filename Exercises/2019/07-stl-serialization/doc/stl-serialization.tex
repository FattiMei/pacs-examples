\documentclass{beamer}
\usetheme{default}

\setbeamercovered{transparent}
\AtBeginSubsection[\inserttocsection]
{
  \begin{frame}<beamer>{Outline}
    \tableofcontents[currentsection,currentsubsection]
  \end{frame}
}

\AtBeginSection[]
{
  \begin{frame}<beamer>{Outline}
    \tableofcontents[currentsection]
  \end{frame}
}

\usepackage[]{listings}
\lstset{language=C++,
basicstyle=\ttfamily,
keywordstyle=\color{blue}\ttfamily,
stringstyle=\color{red}\ttfamily,
commentstyle=\color{green}\ttfamily,
morecomment=[l][\color{magenta}]{\#}}

\usecolortheme{default}
\begin{document}


\title{Serialization of STL Containers}
\date{\today}

\begin{frame}
\maketitle
\end{frame}

\begin{frame}
\frametitle{Serialization}

\begin{itemize}
\item \alert{Serialization}is the process of translating data structures or object state into a format that can be 
\begin{itemize} 
\item \alert{stored}, for example, in a file or memory buffer.
\item \alert{transmitted}, for example, across a network connection link
\item \alert{reconstructed} possibly in a different computer environment.
\end{itemize}
\item When the resulting series of bits is reread according to the serialization format, it can be used to create a semantically identical clone of the original object
\item For many complex objects this process is not straightforward
\item Serialization of object-oriented objects does not include any of their associated methods
\end{itemize}
\end{frame}


\begin{frame}
\frametitle{Doxygen}

\begin{itemize}
\item \alert{Doxygen} is the \textit{de facto} standard tool for generating documentation from annotated C++ sources. 
\item Doxygen can:
\begin{itemize}
\item generate an on-line documentation browser (in HTML) and/or an off-line reference manual (in $\mbox{\LaTeX}$) from a set of documented source files. 
\item extract the code structure from undocumented source files. This is very useful to quickly find your way in large source distributions. Doxygen can also \alert{visualize} the relations between the various elements.
\item create normal documentation (as I did for the doxygen user manual and web-site).
\end{itemize}
\end{itemize}

\end{frame}

\begin{frame}[fragile]
\frametitle{Doxygen}
Use the \texttt{doxygen} command to:
\begin{enumerate}
\item generate a template configuration file:\\
    \verb|doxygen [-s] -g [configName]|

\item generate documentation using an existing configuration file:\\
    \verb|doxygen [configName]|
\end{enumerate}
\end{frame}

\begin{frame}[fragile]
\frametitle{Exercise}
\begin{enumerate}
\item Create functions (actually function templates) to read and write 
\begin{enumerate}
\item \lstinline{std::vector<T>}, where \lstinline{T} is a trivial type
\item \lstinline{std::map<F,S>}, where \lstinline{F}, \lstinline{S} are trivial types
\item \lstinline{std::vector<std::map<F,S>>}, where \lstinline{F}, \lstinline{S} are trivial types
\end{enumerate}
\item T is \alert{trivial type} if it is a scalar type, 
a trivially copyable class with a trivial default constructor, 
or array of such type/class
\item \lstinline{type_traits} defines a compile-time template-based interface to query or modify the properties of types
\item Use doxygen to get a description of the function templates headers to implement
\end{enumerate}
\end{frame}

\end{document}
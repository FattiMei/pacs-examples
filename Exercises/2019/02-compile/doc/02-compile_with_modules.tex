\documentclass[9pt]{beamer}
\usetheme{default}
\setbeamercovered{transparent}
\AtBeginSubsection[\inserttocsection]
{
  \begin{frame}<beamer>{Outline}
    \tableofcontents[currentsection,currentsubsection]
  \end{frame}
}

\AtBeginSection[]
{
  \begin{frame}<beamer>{Outline}
    \tableofcontents[currentsection]
  \end{frame}
}
\begin{document}

\title{Linking to shared libraries distributed with the environmental modules system}
\date{17.03.2017}

\begin{frame}
\maketitle
\end{frame}

\section{Some Useful things to remember about shared libraries}

\begin{frame}
\frametitle{Using shared libraries}
We need to distinguish the bare \emph{use} of a shared library and how
to develop a library. The latter aspect will be covered in a later lecture.
\medskip

Furthermore, in the latter case we need to distinguish between the
\emph{development} phase (i.e. the library is not yet available to the
``general public'') and the \emph{release} phase (i.e. when you make your
library available to others).
\end{frame}


\begin{frame}
  \frametitle{Shared libraries in Linux/Unix} In
  discussing shared libraries we need first to distinguish between the
  linking phase of the compilation process and the loading of the
  executable. 
\medskip

We will treat the linking aspects later on, yet to fully understand
how the handling of shared libraries (and the selection of the correct
version) works on Linux (and generally in POSIX-Unix systems) we need
to understand the difference between \emph{version} and \emph{release}
and the corresponding naming scheme.
\end{frame}

\begin{frame}
  \frametitle{Versions and releases} 
The \emph{version} is a symbol
  (typically a number) by which we indicate a set of instances of a
  library with \alert{a common public interface and functionality}.
  \smallskip

  Within a version, one may have several releases, typically indicated
  by one or more numbers (mayor and minor or bug-fix). A new release is 
  issue to fix bugs  or improve of a library \alert{without
    changes in its public interface}. So a code linked against version 1,
  release 1 of a library should work (in principle) when you update
  the library to version 1, release 2.
\smallskip

  Normally version and releases are separated by a dot in the library name:
  \texttt{libfftw3.so.3.2.4} is version 3, release 2.4 of the
  \texttt{fftw3} library (The Fastest Fourier Transform in the West).
\end{frame}

\begin{frame}
\frametitle{Naming scheme of shared libraries (Linux/Unix)}
We give some nomenclature used when describing a shared library

\begin{itemize}
\item \alert{Link name}. It's the name used in the linking stage when
  you use the \texttt{-l\textcolor{red}{mylib}} option.  It is of the
  form \texttt{lib\textcolor{red}{mylib}.so}. The normal search rules
  apply. Remember that it is also possible to give the full path of
  the library instead of the \texttt{-l} option.
\item \alert{soname} (shared object name).  It's the name looked after
  by the \emph{loader}.  Normally it is formed by the link name
  followed by the version.  For instance
  \texttt{\textcolor{red}{libfftw3.so.3}}. It is \emph{fully
    qualified} if it contains the full path of the library.
\item \alert{real name}. It's the name of the actual file that stores the library. 
  For instance \texttt{\textcolor{red}{libfftw3.so.3.2.4}}
\end{itemize}
\end{frame}


\begin{frame}
  \frametitle{How does it work?}  The command
  \texttt{ldd} lists the shared libraries used by an executable object.
On my computer:
\begin{semiverbatim}
> ldd /usr/bin/octave | grep fftw3.so \newline
libfftw3.so.3 => /usr/lib/libfftw3.so.3 (...)
\end{semiverbatim}
It means that the version of octave I have has been linked (by its
developers) against version $3$ of the \texttt{libfftw3} library, 
as indicated by the \texttt{soname}.

The \alert{loader} searches the occurrence of this library in special
directories (we will discuss about it later) and has indeed found
\texttt{/usr/lib/libfftw3.so.3} (full qualified name). This is the library
used if I launch \texttt{octave} in my computer.  \smallskip

Which release? Well, lets take a closer look at the file
\begin{semiverbatim}
> ls -l /usr/lib/libfftw3.so.3\newline
/usr/lib/libfftw3.so.3 -> libfftw3.so.3.2.4
\end{semiverbatim}
I am in fact using release \texttt{2.4} of version \texttt{3}.
\end{frame}

\begin{frame}
  \frametitle{Got it?}  

  The executable (\texttt{octave}) contains the
  information on which shared library to load, including version
  information, (its soname). This part has been taken care by the 
  developers of Octave.
  \smallskip

  When I launch the program the loader looks in special directories,
  among which \texttt{/usr/lib} for a file that matches the
  \texttt{soname}. This file is typically a symbolic link to the real
  file containing the library.  
\medskip

If I have a new release of \texttt{fftw3} version 3, lets say $2.5$,
I just need to place the corresponding file in the \texttt{/usr/lib}
directory, reset the symbolic links and automagically \texttt{octave}
will use the new release (this is what \texttt{apt-get} does when
installing a new library in a Debian/Ubuntu system).

  \smallskip

  No need to recompile anything!
\end{frame}


\begin{frame}
  \frametitle{Another nice thing about shared libraries} 

  A shared library may depend on another shared library. This information may be encoded  when creating the library
  (just as for an executable, we will see it later on)
  For instance
\begin{semiverbatim}
>ldd /usr/lib/libumfpack.so \newline
...\newline
libblas.so.3gf => /usr/lib/libblas.so.3gf
\end{semiverbatim}
The UMFPACK library is linked against version
\texttt{3gf} of the BLAS library (a particular implementation of the
BLAS). \smallskip

This fact helps in avoiding using incorrect version of
libraries. Not only but when creating an executable that needs UMFPACK I have to indicate only
\texttt{-lumfpack}! \alert{Note:} This is not true for static libraries: you have to list all dependencies.
\end{frame}

\begin{frame}
  \frametitle{How to link to a shared library} Let assume we are
  developing a code, for simplicity contained in a single file
  \texttt{main.cpp} (the treatment of more complex situations is
  immediate) that needs to link to a library, for instance the
  \texttt{fftw3} seen before, and we want to use the shared library.
  \smallskip

  We may note that in the \texttt{/usr/lib/} directory we have also a
  file called \texttt{libfftw3.so}, this is the \emph{linker name} of
the library and in fact it is a link to the real library.
\begin{semiverbatim}
/usr/lib/libfftw3.so -> libfftw3.so.3.2.4
\end{semiverbatim}
\medskip This is the file used by the linker to verify the existence
of the library, to control of the symbols required by our code, and
find the \texttt{soname}.
\end{frame}
\begin{frame}
  \frametitle{How to link with a shared library} 
  
It is now sufficient to proceed as usual
\begin{semiverbatim}
g++ -c main.cpp\newline
g++ -o main -L/usr/lib -lfftw3
\end{semiverbatim}
(note that \texttt{-L/usr/lib} is not strictly necessary since the
linker always look in that directory.)

The linker finds \texttt{libfftw3.so}, controls the symbols it
provides and verifies if the library \alert{contains a
  \texttt{soname}} (if not the link name is assumed to be also the
soname).

Indeed \texttt{libfftw3.so} provides a \texttt{soname}. If we wish we
can check it:
\begin{semiverbatim}
> objdump libx.so.1.3 -p | grep SONAME \newline
SONAME   libfftw3.so.3
\end{semiverbatim}
(of course this has been taken care by the library developers).
\end{frame}
\begin{frame}

  Being \texttt{libfftw3.so} a shared library the linker does not
  resolve the symbols by integrating the corresponding code in the
  executable. Instead, it inserts the information about the
  \texttt{soname} of the library:
\smallskip

\begin{semiverbatim}
> ldd main\newline
libfftw3.so.3 => /usr/lib/libfftw3.so.3 (...)
\end{semiverbatim}
\smallskip
The loader can then do its job now!.
\smallskip

In conclusion, linking with a shared library is not more complicated
than linking with a static one.
\medskip

\alert{Remember:} By default if the linker finds both the static and
shared version of a library it gives precedence to the shared
one. If you want to by sure to link with the static version you need to use 
the \texttt{-static} linker option.
\end{frame}

\begin{frame}
  \frametitle{Where does the loader search for shared libraries?}  

  It looks in \texttt{/lib}, \texttt{/usr/lib} and in all the
  directories contained in \texttt{/etc/ld.conf} or in files with
  extension \texttt{conf} contained in the \texttt{/etc/ld.conf.d/}
  directory (so the search strategy is different than that of the
  linker!)  \smallskip

  If I want to permanently add a directory in the search path of the
  loader I need to add it to \texttt{/etc/ld.conf}, or add a conf file
  in the \texttt{/etc/ld.conf.d/} directory with the name of the
  directory, and then \alert{launch \texttt{ldconfig}}).  \smallskip

  The command \texttt{ldconfig} rebuilds the data base of the shared
  libraries and should be called every time one adds a new library (of
  course \texttt{apt-get} does it for you, and moreover
  \texttt{ldconfig} is launched at every boot of the computer).
  \smallskip

  \alert{Note:} all this operations require you act as superuser, for
  instance with the \texttt{sudo} command.
\end{frame}

\begin{frame}
\frametitle{Alternative ways of directing the loader}
\begin{itemize}
\item Setting the environment variable \texttt{LD\_LIBRARY\_PATH}. If
  it contains a comma-separated list of directory names the
  loader will first look for libraries on these directories.
  \begin{semiverbatim}
    export LD\_LIBRARY\_PATH+=:\alert{dir1}:\alert{dir2}
  \end{semiverbatim}
  \item With a special option, \textcolor{blue}{\texttt{-Wl,-rpath=\alert{directory}}}
 during the compilation of the executable, for instance
   \begin{semiverbatim}
     g++ main.cpp -o main\_dev -Wl,-rpath=/opt/lib  -L. -lsmall
  \end{semiverbatim}
 Here the loader will look in \texttt{/opt/lib} before the standard directories. You can use also relative paths.
\item Launching the command \texttt{sudo ldconfig -n \alert{directory}} which adds \texttt{directory} to the loader search path (you need to have superuser privileges). This addition remains valid until the next reboot of the computer. \alert{NOTE: prefer the other alternatives!}

\end{itemize}
\end{frame}
\begin{frame}{How to build a shared library?}
We will dedicate another lecture to this issue, where we will also show how to handle shared libraries and symbols dynamically.
\end{frame}

\section{Toolchains, Environmental Modules, Shared Libraries and Headers}

\begin{frame}
\frametitle{Toolchains}
The toolchain: a set of software which constitutes a close environment for the development of further software

\begin{itemize}
\item 	a C compiler (and related libc)
\item 	a linker
\item basic tools for software development
\item 	basic libraries for scientific development
\end{itemize}	

All the software must be independent from the hosting OS 
(although it must be capable of talking with the specific kernel and with the CPU instruction set); if this is achieved, the software set will be portable on a large variety of computers, provided that the OS and the architecture is the same.

\end{frame}

\begin{frame}
\frametitle{Environmental Modules}

How do I confine my work “inside” a specific toolchain?\\[3mm]

We use the bash enviromental variables to define paths for executables and libraries; the tool to manage these variables is commonly called 
the enviromental module system.\\[3mm]

There are two main module systems, the one we use is based on the lua language and is called Lmod library.\\[3mm]

A module is basically a set of instructions that define a specific environment for a specific software: where are the executables, where are its libraries, where are the header files, … 
If the software is built inside a toolchain, the installation and the environment  will be portable (in the previous sense).\\[3mm]

The specific module is “loaded” and can be “unloaded” as well when not needed\\[3mm]

The module system can define a hierarchy: a module may need another one, and the system automatically handles these dependencies.

\end{frame}

\begin{frame}[fragile]

Initialization of the modulen  system

\begin{verbatim}
$ export $mkPrefix=/u/sw   #(this variable must be explicitely defined)
$ source $mkPrefix/etc/profile

$ module avail
--------------------------- /u/sw/modules/toolchains ---------------------------
   gcc-glibc/4.8	gcc-glibc/4.9	gcc-glibc/5 (D)
\end{verbatim}

Choice of the toolchain
\begin{verbatim}
$ module load gcc-glibc/5
\end{verbatim}

“Inside” a specific toolchain I can find many other softwares
\begin{verbatim}
$module avail
\end{verbatim}
\end{frame}


\begin{frame}[fragile]
Now a specific software package can be selected: usually many other requested modules will be automatically loaded
\begin{verbatim}
$module avail
------------------ /u/sw/pkgs/toolchains/gcc-glibc/5/modules -------------------
   R/3.2.5            		hypre/2.11.0   	qrupdate/1.1.2
   arpack/3.3.0       	lapack/3.6.0   	scalapack/2.0.2
   blacs/1.1          		lis/1.5.65     	scipy/1.11.0
   boost/1.60.0       	matio/1.5.6    	scotch/6.0.4
   cgal/4.8           		metis/5        	suitesparse/4.5.1
   dealii/8.4.1       		mumps/5.0.1    	superlu/5
   eigen/3.2.8        	muparser/2.2.5 	tbb/4.4
   fenics/2016.1.0    	netcdf/4.4.0   	triangle/1.5.0
   fftw/3.3.4         		octave/4.2.0   	trilinos/11.14.3
   getfem/5.0         	openblas/0.2.17	trilinos/12.6.1
   getfem/5.1  	(D)	   p4est/1.1      	trilinos/12.6.3   (D)
   glpk/4.60          		petsc/3.6.3
   hdf5/1.8.16        	qhull/2015.2
   \end{verbatim}

The module system takes care of the integrity of the definitions and does not permit bad cross-coupling (e.g. libraries with an incompatible libc)

unload all the loaded modules
\begin{verbatim}

$module purge
\end{verbatim}
\end{frame}


\begin{frame}[fragile]

The module system takes care of defining the variable LD\_LIBRARY\_PATH so the loader can find newly added libraries

The user is responsible of passing compiler and linker flags when compiling, to do so he/she must know where libraries and headers are located

The module system provides variables to make it easy to find headers and libraries of loaded modules

\begin{verbatim}
$ module load eigen
$ module list
Currently Loaded Modules:
  1) gcc-glibc/5   2) eigen/3.3.3
$ printenv mkEigenInc
/u/sw/pkgs/toolchains/gcc-glibc/5/pkgs/eigen/3.3.3/include/eigen3
$ printenv mkEigenLib # eigen is headers only!
$ 
\end{verbatim}

General naming scheme: 
\begin{itemize}
\item Headers : {\tt ``mk''+``Modulename''+``Inc''}, {\it e.g.} {\tt mkEigenInc}
\item Libraries : {\tt ``mk''+``Modulename''+``Lib''}, {\it e.g.} {\tt mkSuitesparseLib}
\end{itemize}
\end{frame}

\section{Exercises}
\begin{frame}[fragile]
\frametitle{Code used for the exercise}

For the following exercise we will be using the code from the linear algebra example.

The code has been copied here: https://github.com/pacs-course/pacs/tree/master/Exercises/2017/02-compile

\tiny
\begin{verbatim}
$ ls -R
.:
doc  include  src

./doc:
01-compile_with_modules.tex

./include:
COOExtractor.hpp  LinearAlgebraTraits.hpp  bicg.hpp      cgs.hpp    gmres_util.hpp  qmr.hpp
GetPot            MM_readers.hpp           bicgstab.hpp  cheby.hpp  iml++.hpp       thomas.hpp
GetPot.hpp        Utilities.hpp            cg.hpp        gmres.hpp  ir.hpp

./src:
Makefile  Pres_Poisson.mtx  Utilities.cpp  compile.sh  compile_simple.sh  data.pot  test_all.cpp
\end{verbatim}

\end{frame}

\subsection{Exercise 1 : compiler and linker options}
\begin{frame}\frametitle{Exercise 1 : compiler and linker options}
\begin{itemize}
\item Compile {\tt Utilities.cpp} and make it a shared library named {\tt liblinearalgebra.so}\\[5mm]
\item Compile {\tt test\_all.cpp} and make an executable binary linked to {\tt liblinearalgebra.so}
\end{itemize}
\end{frame}

\begin{frame}[fragile]\frametitle{Exercise 1 : solution}
\begin{verbatim}
$ g++ -std=c++11 -c -I../include -I$mkEigenInc -fPIC Utilities.cpp

$ g++ -shared -Wl,-soname,liblinearalgebra.so \
  Utilities.o -o liblinearalgebra.so
  
$ g++ -std=c++11 -c -I../include -I$mkEigenInc \
  -I$mkSuitesparseInc test_all.cpp
  
$ g++ -L$mkSuitesparseLib -lumfpack -L$PWD \
  -llinearalgebra test_all.o -o test_all
  
\end{verbatim}


\end{frame}



\subsection{Exercise 2 : a simple bash script to automate repetitive operations}
\begin{frame}\frametitle{Exercise 2 : a simple bash script to automate repetitive operations}
Create a simple bash script to build the library and executable of Exercise 1 without typing the same commands every time
\end{frame}

\begin{frame}[fragile]\frametitle{Exercise 2 : solution}
\begin{verbatim}
#!/bin/sh

set -x

g++ -std=c++11 -c -I../include -I$mkEigenInc -fPIC Utilities.cpp

g++ -shared -Wl,-soname,liblinearalgebra.so Utilities.o \
    -o liblinearalgebra.so
    
g++ -std=c++11 -c -I../include -I$mkEigenInc \
    -I$mkSuitesparseInc test_all.cpp
    
g++ -L$mkSuitesparseLib -lumfpack -L$PWD -llinearalgebra \
    test_all.o -o test_all

set +x
\end{verbatim}
\end{frame}



\subsection{Exercise 3 : a smart bash script to avoid unnecessary operations}
\begin{frame}\frametitle{Exercise 3 : a smart bash script to avoid unnecessary operations}
Create a more advanced bash script to build the library and executable of Exercise 1 without repeating compilation / linking steps when not necessary
\end{frame}

\begin{frame}[fragile,allowframebreaks]\frametitle{Exercise 3 : solution}
\begin{verbatim}
#!/bin/sh


set -x

function do_distclean {

    do_clean

    if [[ -a test_all ]]
    then
        rm -rf ./*.o ./*.so test_all
    fi
}


function do_clean {
    if [[ -a Utilities.o  || -a test_all.o \
       || -a liblinearalgebra.so ]]
    then
        rm -rf ./*.o ./*.so 
    fi
}


function do_build_library {
    if [[ ! ( -a Utilities.o ) \
       || ( -a Utilities.o && Utilities.cpp -nt Utilities.o) ]]
    then
        g++ -std=c++11 -c -I../include -I$mkEigenInc \
        -fPIC Utilities.cpp
    fi

    if [[ ! ( -a liblinearalgebra.so ) \
       || ( -a liblinearalgebra.so \
            && Utilities.o -nt liblinearalgebra.so) ]]
    then
        g++ -shared -Wl,-soname,liblinearalgebra.so Utilities.o \
        -o liblinearalgebra.so
    fi
}

function do_build_executable {

    if [[ ! ( -a liblinearalgebra.so ) ]]
    then
        do_build_library
    fi
       
    if [[ ! ( -a test_all.o ) \
       || ( -a test_all.o && test_all.cpp -nt test_all.o) ]]
    then
        g++ -std=c++11 -c -I../include -I$mkEigenInc \
        -I$mkSuitesparseInc test_all.cpp
    fi

    if [[ ! ( -a test_all ) \
       || ( -a test_all \
            && test_all.cpp -nt test_all.o) ]]
    then
        g++ -L$mkSuitesparseLib -lumfpack -L$PWD \
        -llinearalgebra test_all.o -o test_all
    fi
}


case $1 in
    distclean*)        
        do_distclean
        ;;
    clean*)        
        do_clean
        ;;
    library*)
        do_build_library
        ;;
    executable*)
        do_build_executable
        ;;
esac

set +x\end{verbatim}
\end{frame}

\subsection{Exercise 4 : a Makefile}
\begin{frame}\frametitle{Exercise 4 : a Makefile}
Create a Makefile that works as the bash script in Exercise 3
\end{frame}

\begin{frame}[fragile,allowframebreaks]\frametitle{Exercise 4 : solution}
\begin{verbatim}
CXXFLAGS = -std=c++11 -O2 -fPIC
CPPFLAGS = -DNDEBUG -I$(mkSuitesparseInc) -I$(mkEigenInc) -I../include
LDFLAGS = -L. -L$(mkSuitesparseLib)
LIBS = -lumfpack -llinearalgebra

.PHONY : all clean distclean

all : test_all

test_all : test_all.o liblinearalgebra.so
	$(CXX) $(LDFLAGS) test_all.o -o test_all $(LIBS)


test_all.o : test_all.cpp
	$(CXX) $(CPPFLAGS) $(CXXFLAGS) -c test_all.cpp

liblinearalgebra.so : Utilities.o
	$(CXX) $(LDFLAGS) -shared -Wl,-soname,liblinearalgebra.so \
	Utilities.o -o liblinearalgebra.so

Utilities.o : Utilities.cpp
	$(CXX) $(CPPFLAGS) $(CXXFLAGS) -c Utilities.cpp

clean :
	$(RM) *.o 

distclean : clean
	$(RM) liblinearalgebra.so test_all
\end{verbatim}
\end{frame}
\end{document}
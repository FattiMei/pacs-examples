\documentclass{article}
\usepackage{amsmath,amssymb}
\begin{document}
This directory contains a set of ulility to define parametric models.
The chosen paradigm is that of having a collection of simple classes that may be composed to 
create the different types of models and solution procedures.
To this purpose generic programming via templates it is of advantage because it allows to create only loosely reltated classes, contrary to 
polymorphism that requires a precise definition of class hyerarchy and careful design of the interface.

In particular, we have investigated polynomial univariate regression, as an example of models of the form
\begin{equation}
m(x;\boldsymbol{\beta})=\sum_{i=0}^{N-1} \beta_i\phi_i(x)
\end{equation}
where the $\beta_i$ are the parameters, and $\phi_i$ basis functions.

In the case of polynomial regression we have chosen $\phi_i=x^i$, but other choices may be possible.

In this frame, the model il linear with respect to the parameters.

We have also created a class for cost functions, at the moment we have implemented only the mean least square error (MSE)

\begin{equation}
J(\mathbf{x};\mathbf{y},\boldsymbol{\beta})=\dfrac{1}{2n}\sum_{j=0}^{n-1}\vert m(x_j;\boldsymbol{\beta})-y_j\vert^2
\end{equation}

If we use a model based linear w.r.t. the parameters we can find the parameters that minimize $J$ give a set of training points $\{\mathbf{x},\mathbf{y}\}$
using QR factorization. This is very effective if the number of parameters is not too high, let say $\le 100$.

One builds
\[
A=\begin{bmatrix}
\phi_0(x_0)&\ldots&\phi_{N-1}(x_{0})\\
&\cdots &\\
\phi_0(x_{n-1})&\ldots&\phi_{N-1}(x_{n-1})\\
\end{bmatrix},
\]
computes its (thin) $QR$ factorization; $A=QR$ and solves the triangular system
\[
R\boldsymbol{\beta}=Q^T\mathbf{y}.
\]

For the QR factorization we have used the tools available in the \texttt{Eigen} library.

But one may also think to minimize the cost function using an optimization algorithms.
We have also implemented that choiche usining the \texttt{CppNumericalSolvers} library. It is available as a git submodule, so if you have not installed the submodules
do
\begin{verbatim}
git submodule update --init --recursive
\end{verbatim}

After having downloaded the submodule, you have first to go into \texttt{Examples/src/LinearAlgebra/CppNumericalSolvers} and do
\begin{verbatim}
make
\end{verbatim}

To interface the code with the \texttt{CppNumericalSolvers} library, we have crated a Proxy to a \texttt{CostFunction} class, so that it provides the right interface.

\paragraph{A note:} If you compile the code without specifying the option \texttt{RELEASE=yes} or \texttt{DEBUG=no} you compiler the code without optimization and with the \texttt{-g} flag
activated to enable possible debugging. In this situation the code produces a verbose output if you use the \texttt{CppNumericalSolver} optimizers. The vrebose output is not generated if you
do
\begin{verbatim}
make RELEASE=yes
\end{verbatim}
or
\begin{verbatim}
make DEBUG=no
\end{verbatim}



\end{document}

\documentclass[10pt,a4paper]{article}
\usepackage[utf8]{inputenc}
\usepackage[T1]{fontenc}
\usepackage[italian]{babel}
\usepackage{amsmath}
\usepackage{amsthm}
\usepackage{amsfonts}
\usepackage{amssymb}
\usepackage{graphicx}
\usepackage{listings}
\usepackage{color}
\usepackage{hyperref}
\usepackage{a4wide}
\definecolor{mygreen}{RGB}{28,172,0} % color values Red, Green, Blue
\definecolor{mylilas}{RGB}{170,55,241}
\setlength{\parindent}{0pt}
\theoremstyle{definition}% default
\newtheorem{algo}{Algorithm}
\author{Luca Formaggia}
\title{A note on synthetic division and zero of polynomials}
\begin{document}
  \lstset{language=c++,%
     %basicstyle=\color{red},
     breaklines=true,%
     morekeywords={matlab2tikz},
     keywordstyle=\color{blue},%
     morekeywords=[2]{1}, keywordstyle=[2]{\color{black}},
     identifierstyle=\color{black},%
     stringstyle=\color{mylilas},
     commentstyle=\color{mygreen},%
     showstringspaces=false,%without this there will be a symbol in the places where there is a space
     numbers=left,%
     numberstyle={\tiny \color{black}},% size of the numbers
     numbersep=9pt, % this defines how far the numbers are from the text
     emph=[1]{for,end,break},emphstyle=[1]\color{red}, %some words to emphasise
     %emph=[2]{word1,word2}, emphstyle=[2]{style},    
 }
 
 \maketitle
 \section{Synthetic division and Horner algorithm}
 Let
 \[
 p_m(x)=\sum_{i=0}^{m} a_i x^i,
 \]
be a (real or complex) polynomial of degree at most $m$. The Synthetic Division or Horner's algoritm reads

\begin{algo}\label{alg:horner}
\emph{Computes $p(z)$}
\begin{lstlisting}
auto horner(a,z)
    b[m-1]=a[m];
    for k=m-2,...0 b[k]+=b[k+1]*z + a[k+1]; 
    return {b[0]*z + a[0],b}; 
\end{lstlisting}
\end{algo}
The algorithm, beside computing $p_m(z)$, returns the coefficients $b_j$, $j=0,\ldots m-1$ of the so called \emph{associated polynomial}
\begin{equation}
\label{eq:associated}
q_{m-1}(x;z)=\sum_{j=0}^{m-1}b_j x^j.
\end{equation}

It is a polynomial of degree at most $m$, and the $z$ in $q_{m-1}(x;z)$ indicates that the coefficients $b_j$ are the results of the synthetic division applied in $x=z$ (see Algorithm~\ref{alg:horner}). 

We have the following important result: the polynomial $q_{m-1}(x;z)$ is the quotient of the division of $p_m$ and $x-z$. Which means that 
\begin{equation}
\label{eq:quotient}
\boxed{p_m(x)=q_{m-1}(x;z)(x-z) + p_m(z),}
\end{equation}
and, consequently,
\begin{equation}
\label{eq:quotientder}
\boxed{\frac{d}{dx}p_m(x)=\frac{d}{dx}q_{m-1}(x;z)(x-z) + q_{m-1}(x;z),}
\end{equation}
by which
\begin{equation}
\label{eq:quotientder}
\boxed{\frac{d}{dx}p_m(z)=q_{m-1}(z;z).}
\end{equation}
Therefore, the synthetic division is a tool to compute not only the value at a point $z$, but also the derivative at that point and the quotient of the division with the factor $x-z$. It provides the basic ingredients of the \textbf{Newton-Horner} algorithm for computing the zeros of a polynomial. We wil describe it in the general form
\begin{algo}
    Given $z\in\mathbb{C}$, a polynomial $p_m(x)$ and a tolerance $\epsilon$, 
    set $q_m(x;z)=p(x)$ for consistency of notation and do:
    \begin{enumerate}
        \item For $i=0,\ldots,m-1$;
        \begin{enumerate}
            \item Using synthetic division compute 
            \[
            \delta=-q_{m-i}(z;z)/q_{m-i}^\prime(z;z)= -q_{m-i}(z;z)/q_{m-i-1}(z;z);
            \]
            \item $z \leftarrow z+\delta$;
            \item if $\vert\delta\vert<\epsilon$, break and $z$ is an approximation of the $i+1$-th zero of $p_m(x)$; 
        \end{enumerate}
    \item End
    \end{enumerate}
In this algorimm $q_{m-i-1}(x;z)$ indicates the polynomial associated to $q_{m-i}(x;z)$, i.e., for $i=0,\ldots, m-1$,
\begin{equation}
\label{eq:associatedgen}
\boxed{q_{m-i}(x;z)=q_{m-i-1}(x;z)(x-z)+q_{m-i}(z;z),}
\end{equation}
where $q_m(x;z)=p(x)$.
\end{algo}
\section{Derivatives}
Leveraging Equations~\eqref{eq:associatedgen} and~\eqref{eq:quotient} we can also compute higher derivatives at the point $z$. Let's first look to the second derivative
\begin{multline}\label{eq:secderexp}
\frac{d^2p_m}{dx^2}(x)=\frac{d^2q_{m-1}}{dx^2}(x;z)(x-z) + \frac{dq_{m-1}}{dx} (x;z) + \frac{dq_{m-1}}{dx} (x;z)=\\ \frac{d^2}{dx^2}q_{m-1}(x;z)(x-z) + 2 \frac{d}{dx} q_{m-1}(x;z),
\end{multline}
Thus,
\begin{equation}\label{eq:secder}
\frac{d^2p_m}{dx^2}(z)=2 \frac{d}{dx} q_{m-1}(z;z),
\end{equation}

Using~\eqref{eq:associatedgen} with $i=1$
\begin{equation}\label{eq:firstderass}
\frac{d}{dx}q_{m-1}(x;z)=\frac{d}{dx}q_{m-2}(x;z)(x-z)+ q_{m-2}(x;z),
\end{equation}
which combined with~\eqref{eq:secder} gives
\begin{equation}\label{eq:secderivative}
\frac{d^2p_m}{dx^2}(z)=2 q_{m-2}(z;z).
\end{equation}
So the second derivative at $z$ is twice the associated polynomial of $q_{m-1}(x;z)$ at the given point. So it can be computed just by using two synthetic divisions.
What about the third derivative? We have, from~\eqref{eq:secderexp}
\begin{multline}\label{eq:secderexp}
\frac{d^3p_m}{dx^3}(x)=\frac{d^3}{dx^3}q_{m-1}(x;z)(x-z) +\frac{d^2}{dx^2}q_{m-1}(x;z) + 2 \frac{d^2}{dx^2} q_{m-1}(x;z)=\\
\frac{d^3}{dx^3}q_{m-1}(x;z)(x-z) +3\frac{d^2}{dx^2}q_{m-1}(x;z).
\end{multline}
Using~\eqref{eq:firstderass}
\begin{equation}
\frac{d^2}{dx^2}q_{m-1}(x;z)=\frac{d^2}{dx^2}q_{m-2}(x;z)(x-z)+2\frac{d}{dx}q_{m-2}(x;z),
\end{equation}
and thus
\begin{equation}\label{eq:thirdder}
\frac{d^3p_m}{dx^3}(x)=
\frac{d^3}{dx^3}q_{m-1}(x;z)(x-z) +3\frac{d^2}{dx^2}q_{m-2}(x;z)(x-z)^2 + 6\frac{d}{dx}q_{m-2}(x;z),
\end{equation}
from which, exploiting again~\eqref{eq:associatedgen}, now with $i=1$, we have
\begin{equation}\label{eq:thirdderz}
\frac{d^3}{dx^3}p_m(z)=6 q_{m-3}(z;z).
\end{equation}

In general, for any $n\le m$, we have
\begin{equation}
\boxed{\frac{d^n}{dx^n}p_m(z)=n! q_{m-n}(z;z).}
\end{equation}

Therefore, all derivatives at a given point $z$ may be computed with a repeated use of synthetic division.

\section{The code in \texttt{polyHolder.hpp}}
The code in \texttt{polyHolder.hpp} contains
\begin{itemize}
    \item The function \lstinline|polyEval()| which uses the basic Holder's algoritm to compute the value of a polynomial at a point;
    \item The class \lstinline|polyHolder| that allows to compute derivatives of a polynomial (given its coefficients) at a given point. It also allows to get the coefficient of the
    derivative, as an alternative.
    \item The function \lstinline|polyRoots()| which computes the roots of a polynomial using the basic \emph{Newton-Holder} procedure.
\end{itemize}



\end{document}